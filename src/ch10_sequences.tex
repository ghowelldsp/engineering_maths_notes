\section{Sequences}
    
    \subsection{Sequences}
        Any function, \(f\), whose input is restricted to positive or negative integer values, \(n\), has an output in
        the form of a sequence of numbers, i.e. \(f(n)\). Accordingly, such a function is called a \textbf{sequence}.

    \subsection{Graphs of Sequences}
        Since the range of a sequence consists of a sequence of discrete numbers, the terms of the sequence, the graph 
        of a sequence will take the form of a collection of isolated points in the Cartesian plane.

    \subsection{Arithmetic Sequences}
        Any sequence defined in the form,
        \begin{equation}
            f(n) = a + nd
        \end{equation}
        where \(n\) is a positive integer, \(a\) is the first term and \(d\) is the common difference, is known as a
        \textbf{arithmetic sequence}, which is also an equation of a straight line. 
        \par \hfill \break
        As an example, a sequence such as \(1, 3, 5, 7, ...\) has a common difference of \(2\) and the first term is
        \(1\) so the sequence may be seen as \(f(n) = 1 + 2n\)

    \subsection{Geometric Sequences}
        Any sequence defined in the form,
        \begin{equation}
            f(n) = Ar^n
        \end{equation}
        where \(n\) is a positive integer, \(A\) is the first term and \(r\) is the common ratio, is known as a
        \textbf{geometric sequence}. If the absolute value of the common ratio is \(|r|<1\) the sequence converges for
        increasing values of \(n\), else is \(|r|>1\) the sequence diverges.
        \par \hfill \break
        As an example, a sequence such as \(16, 8, 4, 2, ...\) has a common ratio of \(1/2\) and the first term is
        \(16\) so the sequence may be seen as \(f(n) = 16(1/2)^n\).

    \subsection{Harmonic Sequences}
        The definition of a harmonic sequence is that its reciprocal must be an arithmetic sequence, and takes the 
        form,
        \begin{equation}
            f(n) = 1 / g(n)
        \end{equation}

    \subsection{Recursive Prescriptions}
        A prescription where each term of a sequence is seen to depend upon another term of the same sequence is called
        a recursive prescription and can make the computing of the sequence more efficient. For example, a sequence of 
        \(3, 3, 12, 30\) could be seen as,
        \begin{equation}
            f(n+2) = 2(f(n+1) + f(n)) \quad \textrm{where} \quad f(1) = 3
        \end{equation}
        which shows that each term of the sequence is the result of the previous two terms of the sequence summed 
        together, then doubled.
        \par \hfill \break
        Given the general form of the current and next value of \textbf{arithmetic} sequence is,
        \begin{equation}
            f(n) = a + nd \quad \textrm{and} \quad f(n+1) = a + (n+1)d = a + nd + d
        \end{equation}
        then the recursive form would be seen as,
        \begin{equation}
            f(n+1) = f(n) + d
        \end{equation}
        \par \hfill \break
        Similarly, the current and next term of a \textbf{geometric} sequence is seen as,
        \begin{equation}
            f(n) = Ar^n \quad \textrm{and} \quad f(n+1) = Ar^{n+1} = Ar^n.r
        \end{equation}
        Hence, in recursive form,
        \begin{equation}
            f(n+1) = r.f(n)
        \end{equation}

    \subsection{Difference Equations}
        The prescription of a sequence such as \(f(n) = 5n - 2\) can be written as the recursive sequence 
        \(f(n+1) = f(n) + 5\). If this is then expressed so that all the unknowns are on the left-hand side, as 
        \(f(n+1)-f(n)=5\), it is of the form of a 1st order \textbf{difference equation}.
        \par \hfill \break
        Any equation of the form,
        \begin{equation}
            a_{n+m} f(n+m)+a_{n+m-1} f(n+m-1)+a_{n+m-2} f(n+m-2)+ ... + a_n f(n) = 0
        \end{equation}
        where the \(a\) coefficients are constants, is an \(m^{th}\) order, homogeneous, constant-coefficient, linear 
        difference equation, also referred to as a \textbf{linear recurrence relation}. The order of a difference 
        equation is the maximum number of terms between any pair. In order to generate the terms of a difference 
        equation, as many terms as the order must be given, i.e. a 3\textsuperscript{rd} order difference equation must
        have 3 terms given.

    \subsection{Solving First Order Difference Equations}
        As has been previously seen, a prescription for an arithmetic or geometric sequence can be expressed as a 
        difference equation. What we wish to achieve in this section is essentially reversing the process, so finding 
        the solution of the difference equation.
        \par \hfill \break
        Any first order difference equation,
        \begin{equation}
            af(n+1) + bf(n) = 0
        \end{equation}
        where the right-hand side is equal to zero is known as a \textbf{homogeneous}. We then assume a general solution 
        of the form \(f(n) = Kw^n\), and insert that into the difference equation, solving to find \(K\) and \(w\).
        \par \hfill \break
        As an example, consider the difference equation \(f(n+1) + 9f(n) = 0\) for \(n \geq 0\) where \(f(0)=6\). Assuming 
        that \(f(n) = Kw^n\), this would then been seen as,
        \begin{equation}
            Kw^{n+1} + 9Kw^n = 0
        \end{equation}
        Factorising out \(Kw^n\), leads to the \textbf{characteristic equation},
        \begin{equation}
            Kw^n (w+9)=0
        \end{equation}
        This means that \(w+9=0\), hence \(w=-9\). The general solution would be \(f(n) = Kw^n = K(-9)^n\), and then 
        given the initial term of \(f(0) = 6\), leads to \(f(0) = K(-9)^0 = K\), so then \(K=6\). Hence the solution 
        to the difference equation is then,
        \begin{equation}
            f(n) = 6(-9)^n
        \end{equation}

    \subsection{Solving Second Order Difference Equations}
        Solving second order homogeneous difference equations of the form,
        \begin{equation}
            af(n+2) + bf(n+1) + cf(n) = 0
            \label{eq:seq_2nd_order_general_eq}
        \end{equation}
        follows the same method as solving first order equations, by assuming the general solution of the equation as 
        \(f(n) = Kw^n\). However, in this case substituting out \(Kw^n\) to form the characteristic equation will lead
        to \(w\) being of a quadratic form, i.e. \(w^2+w+c\), and hence there will then be two outcomes. This leads to 
        the solution of a second order difference equation being of the following form,
        \begin{equation}
            f(n) = K_1 w_1^n + K_2 w_2^n
        \end{equation}
        Since the characteristic equation is a quadratic there will be two values of w yielding the solution,
        \begin{equation}
            f(n) =
            \begin{cases}
                Aw_1^n + Bw_2^n \quad \textrm{if} \quad w_1 \neq w_2 \\
                (A + Bn) w^n \quad \textrm{if} \quad w_1=w_2=w
            \end{cases}
        \end{equation}
        The values of \(A\) and \(B\) are found by applying the two \(w\) terms once found.
        \par \hfill \break
        As an example, considor the second order resursive prescription,
        \begin{equation}
            f(n+1) - 25f(n-1) = 0
        \end{equation}
        with initial conditions \(f(0) = 1\) and \(f(1) = 5\). Given the general solution of \(f(n) = Kw^n\), this can
        then be expressed as,
        \begin{equation}
            Kw^{n+1} - 25Kw^{n-1} = 0 \quad \textrm{and multipling by \(w\) gives} \quad Kw^n w^2 - 25Kw^n = 0
        \end{equation}
        Factoring out \(Kw^n\) leads to the characteristic equation,
        \begin{equation}
            Kw^n(w^2-25) = 0
        \end{equation}
        hence, \(w = \pm 5\). Inserting these values into the general solution of Eq.(\ref{eq:seq_2nd_order_general_eq})
        and using the initial conditions results in \(K_1 = 1\) and \(K_2 = 0\). Hence, the general term can be seen as,
        \begin{equation}
            f(n) = 5^n
        \end{equation}

    \subsection{Limits}
        The limit of a sequence is the number that the output approaches as the input increase to infinity. A sequence 
        with a finite limit is said to be \textbf{convergent} and converge to that limit. A sequence without a finite 
        limit is said to \textbf{diverge}.

    \subsection{Rules of Limits}
        Limits can be manipulated algebraically according to the following rules, firstly assuming:
        \begin{equation}
            \lim_{n\to\infty} f(n) = F \quad \textrm{and} \lim_{n\to\infty} g(n) = G
        \end{equation}
        the \textbf{multiplication} rule is,
        \begin{equation}
            \lim_{n\to\infty} kf(n) = k \lim_{n\to\infty} f(n) = F
        \end{equation}
        then \textbf{sum and differences} rule is,
        \begin{equation}
            \lim_{n\to\infty}{[f(n) \pm g(n)]} = \lim_{n\to\infty} f(n) \pm \lim_{n\to\infty} g(n) = F \pm G
        \end{equation}
        the \textbf{products and quotients} is,
        \begin{equation}
            \lim_{n\to\infty}{[f(n).g(n)]} = \lim_{n\to\infty} f(n) . \lim_{n\to\infty} g(n) = F.G
        \end{equation}
        If the limits of both the numerator and the denominator in a quotient are infinite the limit is called 
        \textbf{indeterminate} and cannot be found without some manipulation of the quotient.
