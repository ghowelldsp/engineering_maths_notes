\section{Series 1}

    If the terms in a sequence, \(f(1)\), \(f(2)\), \(f(3)\), etc are added together to form a sum,
    \begin{equation}
        \sum_{r=1}^n f(r) = f(1) + f(2) + ... + f(n)
    \end{equation}
    Then what is formed is called a \textbf{series}, a sum of terms of a sequence.

    \subsection{Arithmetic Series}
        
        A general arithmetic series is seen as,
        \begin{equation}
            f(n) = a + nd
        \end{equation}
        where \(a\) is the first term and \(d\) is the common difference.

        \par \hfill \break
        The general arithmetic series containing the first \(n\) terms of the arithmetic sequence can be written as,
        \begin{equation}
            \sum_{r=0}^{n-1} (a + rd) = a + (a +d) + (a + 2d) + ...
        \end{equation}
        this can also be written as,
        \begin{equation}
            \sum_{r=0}^{n-1} (a + rd) = (a + [n-1]d) + (a + [n-2]d) + ...
        \end{equation}
        adding these together gives,
        \begin{equation}
            2 \sum_{r=0}^{n-1} (a + rd) = n(2a + [n-1d])
        \end{equation}
        
        \par \hfill \break
        Hence, the generalised arithmetic series can then be seen as,
        \begin{equation}
            \sum_{r=0}^{n-1} (a + rd) = \frac{n}{2}(2a + [n-1d])
        \end{equation}

        \par \hfill \break
        For example, consider the sequence 10, 6, 2, -2, -6, etc, with which the series would then be \(10+6+2-2-6\), 
        etc. This is an arithmetic sequence where the initial term \(a=10\), and the common difference \(d=-4\).  
        Therefore, to find the sum of the first 20 terms, \(n=20\), can be seen as,
        \begin{equation}
            \sum_{r=0}^{n-1} (a + rd) = \sum_{r=0}^{20-1} (10 + r(-4)) = \frac{20}{2}(2(10+[20-1](-4))) = -560 
        \end{equation}

    \subsection{Arithmetic Mean}